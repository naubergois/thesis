%% This is an example first chapter.  You should put chapter/appendix that you
%% write into a separate file, and add a line \include{yourfilename} to
%% main.tex, where `yourfilename.tex' is the name of the chapter/appendix file.
%% You can process specific files by typing their names in at the 
%% \files=
%% prompt when you run the file main.tex through LaTeX.
\chapter{Introduction}

Performance problems such as high response times in software applications have a significant effect on the customer's satisfaction. The explosive growth of the Internet has contributed to the increased need for applications that perform at an appropriate speed. Performance problems are often detected late in the application life cycle, and the later they are discovered, the greater the cost to fix them. The use of stress testing is an increasingly common practice owing to the increasing number of users. In this scenario, the inadequate treatment of a workload generated by concurrent or simultaneous access due to several users can result in highly critical failures and negatively affect the customers perception of the company \cite{Draheim2006b} \cite{Jiang2010} \cite{Molyneaux2009} \cite{Wert2014}. 

\section{Motivation}

Software testing is a expensive and difficult activity. The exponential
growth in the complexity of software makes the cost of testing has only continued to rise. Test case generation can be seen as a search problem. The test adequacy criterion is transformed into a fitness function and a set of solutions in the search
space are evaluated with respect to the fitness function using a metaheuristic search technique. Search-based software testing is the application of metaheuristic search techniques to generate software
tests cases or perform test execution \cite{Afzal2009a} \cite{Gay}.

\section{State of Research on the Stress Testing Automation}


\section{Research Hypothesis}

The stress testing process in the industry still follows a non-automated and ad-hoc model where the designer or tester is responsible for running the tests analyzing the results and deciding which new tests should be performed \cite{Lewis2005}.

In the academic context, a number of studies proving the efficacy of metaheuristics to automate test execution can be found in literature \cite{Afzal2009a}.

Our underlying research hypothesis is as follows:

\begin{mybox}
The use of metaheuristics and hybrid metaheuristics in combination with Q-learning can make it possible to automate the stress test execution process, improving the choice of new test cases for each interaction and finding scenarios that maximize the number of users of the application under test and minimize response time.
\end{mybox}

The purpose of this thesis is to show the validity of this hypothesis through the development of a testbed tool, algorithms that use hybrid metaheuristics and the Q-learning technique and application of validation experiments. This thesis will be useful for load test practitioners and software engineering researchers interested in testing large-scale software systems.


\section{Thesis Overview}

In this section, we present an overview of the works presented in this thesis. This thesis has five main chapters

\section{Thesis Contributions}


\section{Thesis Organization}


