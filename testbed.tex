\chapter{Improving Stress Search Based Testing using Q-Learning and Hybrid Metaheuristic Approach}


\section{Hybrid Approach}

A large number of researchers have recognized the advantages and huge potential of building hybrid metaheuristics. The main motivation for creating hybrid metaheuristics is to exploit the complementary character of different optimization strategies. In fact, choosing an adequate combination of algorithms can be the key to achieving top performance in solving many hard optimization problems \cite{Puchinger2005} \cite{Blum2012}.

The proposed solution makes it possible to create a model that evolves during the test. The proposed solution model uses genetic algorithms, tabu search, and simulated annealing in two different approaches. The study initially investigated the use of these three algorithms. Subsequently, the study will focus in others Population-based and single point search metaheuristics. The first approach uses the three algorithms independently, and the second approach uses the three algorithms collaboratively (hybrid metaheuristic approach).

In the first approach , the algorithms do not share their best individuals among themselves. Each algorithm evolves in a separate way (Fig. \ref{fig:firstaproach}).


\begin{figure}[h]
\centering
\includegraphics[width=0.7\textwidth]{./images/independ.png}
\caption{Use of the algorithms independently}
\label{fig:firstaproach}
\end{figure}



The second approach uses the algorithms in a collaborative mode (hybrid metaheuristic). In this approach, the three algorithms share their best individuals found (Fig. \ref{fig:secondapproach}). The next subsections present details about the used metaheuristic algorithms (Representation, initial population and fitness function). 



\begin{figure}
\includegraphics[width=1\textwidth]{./images/collaborative.png}
\caption{Use of the  algorithms collaboratively}
\label{fig:secondapproach}
\end{figure}

\subsection{Representation}

The solution representation is composed by a linear vector with 23 positions. The first position represents the name of an individual. The second position represents the algorithm (genetic algorithm, simulated annealing, or Tabu search) used by the individual. The third position represents the type of test (load, stress, or performance). The next positions represent 10 scenarios and their numbers of users. Each scenario is an atomic operation: the scenario must log into the application, run the task goal, and undo any changes performed, returning the application to its original state.

Fig. \ref{fig:genomarepresentation} presents the solution representation and an example using the crossover operation. In the example, genotype 1 has the Login scenario with 2 users, the Form scenario with 0 users, and the Search scenario with 3 users. Genotype 2 has the Delete scenario with 10 users, the Search scenario with 0 users, and the Include scenario with 5 users. After the crossover operation, we obtain a genotype with the Login scenario with 2 users, the Search scenario with 0 users, and the Include scenario with 5 users.

\begin{figure}[h]
\centering
\includegraphics[width=1\textwidth]{./images/genomerepresentation1.png}
\caption{Solution representation and crossover example}
\label{fig:genomarepresentation}
\end{figure}

Fig. \ref{fig:neighbourtaby} shows the strategy used by the proposed solution to obtain the representation of the neighbors for the Tabu search and simulated annealing algorithms. The neighbors are obtained by the modification of a single position (scenario or number of users) in the vector.


\begin{figure}[h]
\centering
\includegraphics[width=1\textwidth]{./images/neighbor.png}
\caption{Tabu search and simulated annealing neighbor strategy}
\label{fig:neighbourtaby}
\end{figure}


\subsection{Initial population}

The strategy used by the plugin to instantiate the initial population is to generate 50\% of the individuals randomly, and 50\% of the initial population is distributed in three ranges of values:

\begin{itemize}
\item Thirty percent of the maximum allowed users in the test;
\item Sixty percent of the maximum allowed users in the test; and
\item Ninety percent of the maximum allowed users in the test.
\end{itemize}

The percentages relates to the distribution of the users in the initial test scenarios of the solution. For example, in a hypothetical test with 100 users, the solution will create initial test scenarios with 30, 60 and 90 users.

\subsection{Objective (fitness) function}

The proposed solution was designed to be used with independent testing teams in various situations where the teams have no direct access to the environment where the application under test was installed. Therefore, the IAdapter plugin uses a measurement approach to the definition of the fitness function. The fitness function applied to the IAdapter solution is governed by the following equation:

\begin{equation}
\begin{aligned}
fit=90percentileweigth* 90percentiletime\\
+80percentileweigth*80percentiletime\\+
70percentileweigth*70percentiletime+\\
maxResponseWeigth*maxResponseTime+\\
numberOfUsersWeigth*numberOfUsers-penalty
\end{aligned}
\end{equation}

The proposed solution's fitness function uses a series of manually adjustable user-defined weights (90percentileweight, 80percentileweight,  70percentileweight, maxResponseWeight, and numberOfUsersWeight). These weights make it possible to customize the search plugin's functionality. A penalty is applied when an application under test takes a longer time to respond than the level of service. The penalty is calculated by the follow equation:

\begin{equation}
\begin{aligned}
penalty=100 * \Delta \\
\Delta=(t_{Current Response Time} - t_{Maximum Response Time Expected})\\
\end{aligned}
\end{equation}

\section{Hybrid Metaheuristic with Q-Learning Approach}

The HybridQ algorithm uses the GA, SA and Tabu Search algorithms in a collaborative approach in conjunction with Q-Learning technique. The biggest difference between the Hybrid and HybridQ algorithms is the application of a series of modifications on individuals based on the Q-Learning algorithm before each generation.

\begin{figure}[h!]
\center
\includegraphics[width=1\textwidth]{./images/qhybrid.png}
\caption{Hybrid Metaheuristic with Q-Learning Approach}
\label{fig:hybridq}
\end{figure}

Figure \ref{fig:mdphybridq} shows the proposed MDP model for HybridQ. The model has three main states based on response time. A test may have a response time greater than 1.2 times the maximum response time alloowed, between 0.8 and 1.2 times the maximum response time allowed or less than 0.8 times the maximum response time allowed. A test receives a positive reward when an action increases fitnesse value and a negative reward when an action reduces the fitnesse value. The possible actions in MDP are the change of one of the test scenarios and increase or decrease of the number of users.

\begin{figure}[h!]
\center
\includegraphics[width=0.7\textwidth]{./images/mdp3.png}
\caption{Markov Decision Process used by HybridQ}
\label{fig:mdphybridq}
\end{figure}


Unlike the traditional approach, The update of Q values for each action also occurs in the explotation phase. The exploit phase ends when no value of Q it is equals zero for a state, ie, unlike the traditional approach an agent belonging to one state may be in the exploration phase while another agent may be in the explotation phase. The table presents an hyphotetical Q-values for a test. In the table \ref{pab:mdp}, it can be observed that the agents in the Service Level state are in the exploitation phase because there is no other value of Q equals to zero.


% Please add the following required packages to your document preamble:
% \usepackage[table,xcdraw]{xcolor}
% If you use beamer only pass "xcolor=table" option, i.e. \documentclass[xcolor=table]{beamer}
\begin{table}[]
\centering
\caption{Hypothetical MDP Q-values }
\label{pab:mdp}
\begin{tabular}{lll}
\rowcolor[HTML]{C0C0C0} 
\textbf{Above Service Level}  & \textbf{Scenario 1} & \textbf{Scenario 2} \\
Increment Users               & 0.2                 & 0.0                 \\
Reduce Users                  & 0.1                 & 0.2                 \\
Phase                         & Exploration         & Exploration         \\
\rowcolor[HTML]{C0C0C0} 
\textbf{Service Level}        & \textbf{Scenario 1} & \textbf{Scenario 2} \\
Increment Users               & 0.2                 & 0.11                \\
Reduce Users                  & 0.1                 & -0.2                \\
\rowcolor[HTML]{F8FF00} 
Phase                         & Explotation         & Explotation         \\
\rowcolor[HTML]{C0C0C0} 
\textbf{Bellow Service Level} & \textbf{Scenario 1} & \textbf{Scenario 2} \\
Increment Users               & 0.0                 & 0.2                 \\
Reduce Users                  & 0.1                 & 0.0                 \\
Phase                         & Exploration         & Exploration        
\end{tabular}
\end{table}

The Fig. \ref{fig:neighservice} presents how one of the neighbors of a test is generated using Q-Learning. The solution uses a service called Q-Neighborhood Service to generate the neighbor from the action that has the highest value of Q.


\begin{figure}[h]
\center
\includegraphics[width=0.8\textwidth]{./images/q-neighborservice.png}
\caption{HybridQ NeighborHood Service}
\label{fig:neighservice}
\end{figure}


\section{IAdapter}

IAdapter is a JMeter plugin designed to perform search-based stress tests.  The plugin is available on www.iadapter.org.  The IAdapter plugin implements the solution proposed in Section 5. The next subsections present details about the Apache JMeter tool, the IAdapter Life Cycle and the IAdapter Components. The IAdapter plugin provides three main components: WorkLoadThreadGroup, WorkLoadSaver, and WorkLoadController.

The Fig. \ref{fig:iadapterarchitecture} show the IAdapter architecture. All metaheuristic class implements the interface IAlgorithm. Test scenarios  and test results are stored in a Mysql database. GeneticAlgorithm class uses a framework named JGAP to implement Genetic Algorithms.

\begin{figure}[h]
\center
\includegraphics[width=0.8\textwidth]{./images/iadapter1.png}
\caption{IAdapter architecture}
\label{fig:iadapterarchitecture}
\end{figure}

The WorkLoadThreadGroup class is the Load Injection and Test Management modules, responsible to generate the initial population and uses the JMeter Engine to realize requests to server under test. 

\subsection{IAdapter Life Cycle}
 
Fig. \ref{fig:iadapterlifecycle} presents the IAdapter Life Cycle. The main difference between IAdapter and JMeter tool is that the IAdapter provide an automated test execution where the new test scenarios are choosen by the test tool.  In a test with JMeter, the tests scenarios are usually chosen by a test designer.

\begin{figure}[h]
\centering
\includegraphics[width=0.8\textwidth]{./images/lifecycle2.png}
\caption{IAdapter life cycle}
\label{fig:iadapterlifecycle}
\end{figure}

\subsection{IAdapter Components}
 
WorkLoadThreadGroup is a component that creates an initial population and configures the algorithms used in IAdapter. Fig. \ref{fig:tela1iadapter} presents the main screen of the WorkLoadThreadGroup component. The component has a name \ding{202}, a set of configuration tabs \ding{203}, a list of individuals by generation \ding{204}, a button to generate an initial population \ding{205}, and a button to export the results \ding{206}.

\begin{figure}[h]
\centering
\includegraphics[width=0.8\textwidth]{./images/tela1iadapter.png}
\caption{WorkLoadThreadGroup component}
\label{fig:tela1iadapter}
\end{figure}

WorkLoadThreadGroup component uses the GeneticAlgorithm, TabuSearch and SimulateAnnealing classes.  The WorkLoadSaver component is responsible for saving all data in the database. The operation of the component only requires its inclusion in the test script.

WorkLoadController represents a scenario of the test. All actions necessary to test an application should be included in this component. All instances of the component need to login into the application under test and bring the application back to its original state.


\subsection{IAdapter Testbed Tool}

In this subsection, We devise a new testbed that has the ability to reproduce different types of web workloads.  The proposed solution extends a tool named IAdapter to create a testbed tool to validade load, performance and stress search based test approaches \cite{Gois2016}. This new testbed must accomplish three main goals. First, it must reproduce a workload by using an antipattern implementation. Second, it must be able to provide metrics with the aim of being used for research evaluation studies. Finally, it should be extensible, allowing the creation of new metaheuristic approaches.

The testbed tool proposed consists of four main modules.  Figure \ref{fig:testbedarch} presents the main architecture of the Testbed solution proposed. The emulator module provides workloads to the Test module. The Test module uses a class loader to find all classes that extends AbstractAlgorithm in the classpath and run all workloads with each metaheuristic found. The Test Scenario library provides the scenario representation used by the metaheuristics and store the testbed results in a database. The Operation services are responsible for finding neighbors of some workload provided as a parameter and perform crossover operations.

\begin{figure}[h]
\centering
\includegraphics[width=1\textwidth]{./images/testbedarch.png}
\caption{Testbed main architecture.}
\label{fig:testbedarch}
\end{figure} 

\begin{figure}[h]
\centering
\includegraphics[width=1\textwidth]{./images/myheuristic.png}
\caption{Test Module class diagram.}
\label{fig:heuristicclassdiagram}
\end{figure} 



\subsection{Test Module}

The Test Module (Figure \ref{fig:testbedarch}  -\ding{202}) is responsible for for the loading of all classes that extend AbstractAlgorithm in the classpath and perform the tests under the application. Figure \ref{fig:heuristicclassdiagram} shows the  class diagram for custom and provided heuristics. All heuristic classes extends the class AbstractAlgorithm. The heuristics receives  as input a  list of workloads (Figure \ref{fig:heuristicclassdiagram}  -\ding{203}) and must return a list of output workloads (the individuals selected for the next generation)  (Figure \ref{fig:heuristicclassdiagram}  -\ding{204}). Each workload represent an individual in the search space. Figure. \ref{fig:step2} presents the Test Module life cycle. Given an initial population (Figure \ref{fig:step2}  -\ding{202}),  a metaheurist select a new set of workloads based on an objective function (Figure \ref{fig:step2}  -\ding{203}). The choosen metaheurist generate a new set of individuals based on crossover or neighborhood operators (Figure \ref{fig:step2}  -\ding{204}).  JMeterEngine run each workload (Figure \ref{fig:step2}  -\ding{206}) and the choosen metaheuristic obtain a fitness value for each workload based on some objective function  (Figure \ref{fig:step2}  -\ding{207}). Each Metaheuristic could define your own objective function. After all these steps the cycle begins until the maximum number of generations it is reached (Figure \ref{fig:step2}  -\ding{208}). 

\begin{figure}[h]

\centering
\includegraphics[width=1\textwidth]{./images/step2.png}
\caption{Test module life cycle.}
\label{fig:step2}

\end{figure} 



\subsection{Emulator Module}

The Emulator Module is responsible for implementing and providing successful scenarios and the most common performance antipatterns (Figure \ref{fig:testbedarch}  -\ding{203}). All classes must extend the AbstractJavaSamplerClient class or use JUnit 4. The AbstractJavaSamplerClient class allows the creation of a JMeter Java Request. Figure \ref{fig:emulator} presents the main features of the emulator module. The module implements 2 happy scenarios (Figure \ref{fig:emulator}  -\ding{203}) and  4 antipatterns test scenarios (Figure \ref{fig:emulator}  -\ding{202}), in its first version. The Mock Layer provides emulated databases and components for the test scenarios. The Mock Layer use the Mockito and PowerMocks frameworks (Figure \ref{fig:emulator}  -\ding{204}). 

\begin{figure}[h]
\centering
\includegraphics[width=1\textwidth]{./images/emulator.png}
\caption{Emulator module}
\label{fig:emulator}
\end{figure}  

\subsection{Test Scenario Library}

This modules provides a common representation for all workloads (Figure \ref{fig:testbedarch}  -\ding{205}). Each workload is composed by a linear vector with 21 positions (Figure \ref{fig:solution}  -\ding{202}). The first position represents an metadata with the name of an individual. The next positions represent 10 scenarios and their numbers of users (Figure \ref{fig:solution}  -\ding{203}). Each scenario is an atomic operation: the scenario must log into the application, run the task goal, and undo any changes performed, returning the application to its original state. 

\subsection{Operation services}

The services are responsible for performing some operations performed by metaheuristics. Figure. \ref{fig:solution} presents the solution representation and an example using the crossover operation. In the example, solution 1 (Figure \ref{fig:solution}  -\ding{204}) has the Login scenario with 2 users, the Search scenario with 4 users, Include scenario with 1 user and the Delete scenario with 2 users.  After the crossover operation with solution 2 (Figure \ref{fig:solution}  -\ding{205}), We obtain a solution with the Login scenario with 2 users, the Search scenario with 4 users, the Update scenario with 3 users and the Include scenario with 5 users (Figure \ref{fig:solution}  -\ding{206}). Figure. \ref{fig:solution} -\ding{207} shows the strategy used by the proposed solution to  obtain the neighbors for the Tabu search and simulated annealing algorithms. The neighbors are obtained by the modification of a single position (scenario or number of users) in the vector.


