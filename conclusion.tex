\chapter{Conclusion}

In this thesis we dealt with the use of hybrid metaheuristics and Q-Learning in Stress Testing. This thesis presented an hybrid and an hybrid with Q-Learning metaheuristic approaches that combines genetic algorithms, simulated annealing, and tabu search algorithms in stress tests. A tool named IAdapter (github.com/naubergois/newiadapter), a JMeter plugin for performing search-based load tests, was developed. Six experiments were conducted to validate the proposed approach. The first experiment was performed on an emulated component. The second and third experiments are conducted in an testbed developed appplication. The fourth  experiment was performed using an installed Moodle application. The fifth and sixth experiments are performned using an installed JPetStore application.

IAdapter Testbed is an open-source facility that provides software tools for search based test research. The testbed tool emulates test scenarios in a controled environment using mock objects and implementing performance antipatterns.

The main contributions of this research are as follows: The presentation of a hybrid metaheuristic using Q-learning  approach for use in stress tests; the development of a Testbed tool the development of a JMeter plugin  for search-based tests and  the automation of the stress test execution process.  

\section{Achievements}

Two experiments were performed to validate the hybrid metaheuristic,two experiments were conducted to validate the testbed tool and two experiments where conducted to validate the hybridQ metaheuristic. The experiments uses genetic, algorithms, tabu search and simulated annealing. 
 
In the first experiment, the signed-rank Wilcoxon non-parametrical procedure was used for comparing the results. The significant level adopted was 0.05. The procedure showed that there was a significant improvement in the results with the Hybrid Metaheuristic approach.

The second and third experiments ran for 17 generations. The experiments used an initial population of 4 individuals by metaheuristic. All tests in the experiment were conducted without the need of a tester, automating the execution of stress tests with the JMeter tool. In both experiments the hybridQ metaheuristic returned individuals with higher fitness scores. However, the Hybrid metaheuristic made twice as many requests than Tabu Search to overcome it. The SA algorithm obtained the worst fitness values. The algorithm initially used a scenario with an antipattern and found neighbors that still using the antipatterns over the 17 generations of the experiment.

In the second experiment the metaheuristics converged to scenarios with an happy path, excluding the scenarios with the use of an antipatterns.The first individual has 153 users on Happy Scenario 2, 16 users on Happy Scenario 1 and a response time of 13 seconds. None of the four best individuals has one of the antipatterns used in the experiment.


In the third experiment,  the metaheuristics converged to scenarios with an happy path and Tower Babel antipattern, excluding the scenarios with Unbalanced Processing antipattern.  The individual with best fitness value has 121 users on Happy Scenario 2, 171 users on Happy Scenario 1 and a response time of 11 seconds. None of the four best individuals has one of the antipatterns used in the experiment.


In the fourth experiment, the whole process of stress and performance tests, which took 3 days and about 1800 executions, was carried out without the need for monitoring by a test designer. The tool automatically selected the next scenarios to be run up to the limit of six generations previously established. The small number of samples of the experiment is insufficient to give a statistical significance to the results of the Wilcoxon procedure. However, it is noted that, in four of six generations, the collaborative approach presented the best values. The experiment succeeded in finding 29 individuals whose maximum time expected by the application was obtained. 

\section{Open Issues and future works}

There is a range of future improvements in the proposed approach. Also as a typical search strategy, it is difficult to ensure that the execution times generated in the experiments represents global optimum. More experimentation is also required to determine the
most appropriate and robust parameters. Lastly, there is a need for an adequate termination criterion to stop the search process.


Among the future works of the research, the use of new combinatorial optimization algorithms such as very large-scale neighborhood search is one that we can highlight. 
