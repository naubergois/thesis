\chapter{Conclusion}

In this thesis we dealt with the use of hybrid metaheuristics and Q-Learning in Stress Testing.

This thesis presented a hybrid metaheuristic approach that combines genetic algorithms, simulated annealing, and tabu search algorithms in stress tests. A tool named IAdapter (github.com/naubergois/newiadapter), a JMeter plugin for performing search-based load tests, was developed. Two experiments were conducted to validate the proposed approach. The first experiment was performed on an emulated component, and the second one was performed using an installed Moodle application.

IAdapter Testbed is an open-source facility that provides software tools for search based test research. The testbed tool emulates test scenarios in a controled environment using mock objects and implementing performance antipatterns.

The main contributions of this research are as follows: The presentation of a hybrid metaheuristic approach for use in stress tests; the development of a Testbed tool the development of a JMeter plugin  for search-based tests and  the automation of the stress test execution process.  

\section{Achievements}

Four experiments were performed to validate the hybrid metaheuristic and two experiments were conducted to validate the Testbed tool. The experiments uses genetic, algorithms, tabu search, simulated annealing and the hybrid approach. 
 
The first experiment was performed on an emulated component, and the second experiment was performed using an installed Moodle application.  The collaborative approach obtained better fit values in both experiments. In the first experiment, the signed-rank Wilcoxon non-parametrical procedure was used for comparing the results. The significant level adopted was 0.05. The procedure showed that there was a significant improvement in the results with the Hybrid Metaheuristic approach.

The second and thrid experiments ran for 17 generations. The experiments used an initial population of 4 individuals by metaheuristic. All tests in the experiment were conducted without the need of a tester, automating the execution of stress tests with the JMeter tool. In both experiments the hybrid metaheuristic returned individuals with higher fitness scores. However, the Hybrid metaheuristic made twice as many requests than Tabu Search to overcome it. The SA algorithm obtained the worst fitness values. The algorithm initially used a scenario with an antipattern and found neighbors that still using the antipatterns over the 17 generations of the experiment.

In the second experiment the metaheuristics converged to scenarios with an happy path, excluding the scenarios with the use of an antipatterns. The individual with best fitness value has 64 users on Happy Scenario 2, 81 users on Happy Scenario 1 and a response time of 12 seconds. None of the best individuals has one of the antipatterns used in the experiment.


In the third experiment,  the metaheuristics converged to scenarios with an happy path and Tower Babel antipattern, excluding the scenarios with Unbalanced Processing antipattern. The individual with best fitness value has 72 users on Happy Scenario 2, 30 users on Happy Scenario 1, 46 user with the antipattern Tower Babel and a response time of 11 seconds. Future works include the use of new antipatterns and more experiments with the use of the antipattern Tower Babel.

In the fourth experiment, the whole process of stress and performance tests, which took 3 days and about 1800 executions, was carried out without the need for monitoring by a test designer. The tool automatically selected the next scenarios to be run up to the limit of six generations previously established. 

\section{Open Issues and future works}

There is a range of future improvements in the proposed approach. Also as a typical search strategy, it is difficult to ensure that the execution times generated in the experiments represents global optimum. More experimentation is also required to determine the
most appropriate and robust parameters. Lastly, there is a need for an adequate termination criterion to stop the search process.


Among the future works of the research, the use of new combinatorial optimization algorithms such as very large-scale neighborhood search is one that we can highlight. 
